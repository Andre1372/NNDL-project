\documentclass[12pt,a4paper]{article}

% Pacchetti
\usepackage[utf8]{inputenc}
\usepackage[italian]{babel}
\usepackage{amsmath}
\usepackage{amssymb}
\usepackage{graphicx}
\usepackage{hyperref}
\usepackage{listings}
\usepackage{xcolor}

% Configurazione listings per codice Python
\lstset{
    language=Python,
    basicstyle=\ttfamily\small,
    keywordstyle=\color{blue},
    commentstyle=\color{green!60!black},
    stringstyle=\color{red},
    numbers=left,
    numberstyle=\tiny\color{gray},
    frame=single,
    breaklines=true,
    showstringspaces=false
}

% Informazioni documento
\title{Progetto di Neural Networks and Deep Learning}
\author{Nome Cognome}
\date{\today}

\begin{document}

\maketitle

\begin{abstract}
Questo documento descrive il progetto sviluppato per il corso di Neural Networks and Deep Learning.
\end{abstract}

\tableofcontents
\newpage

\section{Introduzione}

Descrizione del problema affrontato e degli obiettivi del progetto.

\section{Dataset}

Descrizione del dataset utilizzato, incluse le sue caratteristiche principali e le operazioni di preprocessing effettuate.

\section{Architettura del Modello}

Descrizione dettagliata dell'architettura della rete neurale implementata, inclusi i layer utilizzati e le motivazioni delle scelte progettuali.

\section{Training}

Descrizione del processo di training: funzione di loss, ottimizzatore, learning rate, batch size, numero di epochs, ecc.

\section{Risultati}

Presentazione dei risultati ottenuti, con grafici e metriche di valutazione.

\subsection{Metriche di Performance}

\subsection{Visualizzazioni}

\section{Conclusioni}

Discussione dei risultati ottenuti, limitazioni del lavoro svolto e possibili sviluppi futuri.

\section*{Riferimenti}
\bibliographystyle{plain}
% \bibliography{references}

\end{document}
